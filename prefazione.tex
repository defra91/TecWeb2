\section*{Prefazione}

Risulta evidente come nell'ultimo decennio i metodi reperimento dell'informazione abbiano subito un cambiamento radicale. Oggigiorno reperire informazioni, conoscenza e quant'altro dal web rappresenta una \textit{condicio sine qua non}. Parlare di questo fenomeno risulterebbe noioso, in quanto è un tema scontato e già ampiamente discusso e documentato. La tecnologia non solo ci accompagna quotidianamente, ma è diventata parte integrante di noi stessi e del nostro modo di vivere; uno stile di vita in continua evoluzione, nel quale il rapporto uomo-macchina è sempre più stretto, basti immaginare alla percentuale di tempo che passiamo in compagnia del nostro smartphone o del nostro tablet.
\linebreak
\linebreak
Chiunque oggigiorno dovrebbe avere il sacrosanto diritto di \textbf{avere l'informazione a portata di mano} ed ottenere ciò che cerca \textbf{nel modo più facile e intuitivo possibile}. Il web è stato sviluppato esattamente con questo intento, anche se nella realtà attuale reperire informazioni è solamente una delle tante tipologie di richieste da parte degli utenti. In questa relazione ci occuperemo solamente del primo aspetto. Questo tema sembrerebbe a prima vista banale e poco interessante, ma se ci fermiamo per un'istante e pensiamo a quanti siti web nei quali navighiamo ogni giorno soddisfano realmente queste due necessità credo che dovremmo fare una bella scrematura.
\linebreak
\linebreak
Il web è in continua evoluzione. Un sito web sviluppato oggi in poco tempo diviene obsoleto, perchè cambiano i canoni di progettazione, che sono strettamente correlati alle crescenti e mutabili necessità degli utenti. E queste necessità sono sempre maggiori: gli utenti sono impazienti, vogliono pagine che si caricano velocemente, ben indicizzate dai motori di ricerca e, soprattuto, vogliono \textbf{trovare quel che cercano} in modo rapido. Se la pagina non riesce a soddisfare questo l'utente si dirigerà altrove. Ecco che in quest'ottica è necessario introdurre il termine \textbf{usabilità}, che viene definita come segue:

\begin{center}

\textit{``Il grado in cui un prodotto può essere usato da particolari utenti per raggiungere certi obiettivi con efficacia, efficienza, soddisfazione in uno specifico contesto d’uso''}\footnote{Definizione secondo ISO 9241-11:1998}

\end{center} 

Il contesto d'uso della presente relazione è chiaramente il web. Diamo ora per correttezza una definizione formale di alcuni termini:

\begin{itemize}
\item \textbf{Efficacia} indica la precisione e la completezza con cui gli utenti raggiungono il loro obiettivo;
\item \textbf{Efficienza} indica le risorse impiegate in relazione alla precisione e alla completezza cui gli utenti raggiungono il loro obiettivo;
\item \textbf{Soddisfazione} è la libertà dal \textit{disagio} e attitidine positiva con cui gli utenti raggiungono i loro obiettivi attraverso l'uso del prodotto.

\end{itemize}

Conoscere il significato di questi termini è essenziale per poter anche solo pensare di pubblicare un sito web. Il punto di ottimo è chiaramente la massima efficienza unito alla massima efficacia. Ciò è difficilmente raggiungibile e realisticamente improbabile, ma costituisce comunque un buon punto di riferimento.
\linebreak
\linebreak
Sviluppare un sito web risulta dunque un'operazione estremamente complessa. Se fino a qualche anno fa l'attenzione si poneva altrove e una pagina web costituiva un mezzo secondario di reperimento dell'informazione, al giorno d'oggi per avere successo in un mondo ove la competizione in questo settore è altissima è necessario spendere il giusto tempo per porre l'attenzione a concetti come l'usabilità e l'accessibilità (altro grande argomento). Non voglio in ogni caso essere superficiale, sono consapevole del fatto che l'usabilità ha un costo, perchè aumenta considerevolmente il lavoro in fase di analisi, di progettazione e soprattutto di testing. Ciononostante sono altrettanto convinto delle seguenti argomentazioni:

\begin{itemize}

\item I costi relativi all'usabilità producono a valle del progetto un prodotto enormemente più valido e che può dunque aspirare al successo con una probabilità maggiore;
\item I costi possono essere ammortizzati focalizzando l'attenzione all'usabilità a monte del progetto e dunque progettando il sito web sempre con un occhio di riguardo a quest'aspetto. Rendere usabile un sito web che gode di scarsa o pressochè nulla usabilità ha un costo chiaramente molto più elevato e spesso insostenibile.

\end{itemize}

A mio modo di vedere qualsiasi web designer, front-end developer o progettista di siti web dovrebbe avere nel proprio bagaglio delle buone skills di progettazione che tengano conto dell'usabilità. In caso contrario il loro lavoro sarà chiaramente poco apprezzato e dunque i clienti si rivolgeranno alla folta concorrenza.

Dall'altro lato per poter progettare bene è necessario saper in primo luogo \textbf{analizzare}. Ispezionando le diverse pagine web che incontriamo ogni giorno e giudicarle con spirito critico ci porta ad accumulare esperienza, a farci un'idea di \textit{cosa è giusto} e \textit{cosa è sbagliato}. Chiaramente la materia è opinabile, non esiste uno standard di usabilità, esso evolve in continuazione e ciò che vale oggi non necessariamente è valido anche domani. Un buon sviluppatore web non dovrebbe dunque mai fermarsi, analizzare continuamente, tenersi aggiornato coi tempi. Sapersi rinnovare è una corda di sicurezza, senza la quale nel mondo dello sviluppo web si rischia di cadere.
\linebreak
\linebreak
Questo documento non vuol essere una guida all'usabilità ma un'analisi concreta di usabilità. Risulta allo stesso tempo necessario però definire alcuni concetti fondamentali prima di procedere nell'analisi. Questo è sostanzialmente la ragione per la quale ho scritto questa prefazione. Nei prossimi capitoli che seguiranno analizzerò il sito in questione con spirito critico, senza alcun pregiudizio e con la massima obbiettività. Il \textbf{capitolo 1} si occuperà di definire il sito in questione, studiare la sua storia, la sua struttura e soprattutto il suo scopo. Il \textbf{capitolo 2} inizia con l'analisi delle \textbf{sei W}: \textit{Where}, \textit{Who}, \textit{Why}, \textit{What}, \textit{When}, \textit{How}. Il \textbf{capitolo 3} affronta il problema del tempo e dei timer. Il \textbf{capitolo 4} analizza il contenuto di una pagina interna e come essa si presenta. Il \textbf{capitolo 5} affronta il problema della pubblicità e della sua gestione lungo l'intera navigazione. Il \textbf{capitolo 6} affronta il problema della ricerca interna al sito. Il \textbf{capitolo 7} fornisce infine una valutazione complessiva del sito, indicando gli aspetti positivi e negativi e, laddove opportuno, fornendo una possibile soluzione ai problemi. 

\subsection*{Note sulla licenza}

Il documento è rilasciato sotto licenza Creative Common \textit{by-nc-sa} 4.0. Chiunque può liberamente prendere il seguente documento e redigerne una nuova versione, purchè la redistribuzione non abbia scopi commerciali, venga redistribuita nello stesso modo e sia mantenuto credito all'autore originale. Per maggiori informazioni consultare la seguente pagina: \url{http://creativecommons.org/licenses/by-nc-sa/4.0/}

\begin{center}

\includegraphics[width=50mm]{images/cc.png}

\end{center}